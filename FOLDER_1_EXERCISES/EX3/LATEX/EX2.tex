\documentclass{article}
\usepackage{hyperref}
\begin{document}
\title{Exercise 2}
\maketitle

\section{Introduction}
This exercise implements a simple desktop calculator.
It uses the LR$(1)$ parsing scheme on top of FLEX.
You need not perform the actual computation, but rather say
whether it is a valid arithmetic computation.
For instance, $4+5*6$ is valid, and $6-4*(8$
is \textit{not} valid.

\section{Input}
The input for this exercise is a single text file that contains the arithmetic expression.

\section{Output}
The output is a single text file that should contain a single word:
either OK when the expression is valid, or FAIL otherwise. 

\section{Submission Guidelines}
The code for this exercise resides in the course code repository EX2.
The file Table.c should be the only file you change.
Please submit your exercise in your GitHub repository under
COMPILATION/EX2, and have a makefile there to build a runnable program called calc.
To avoid the pollution of EX2, please remove all *.o files once the target is built.
The next paragraph describes the execution of calc.

\paragraph{Execution parameters}
calc recevies $2$ input file names:
\begin{table}[h]
\centering
\begin{tabular}{ l }
  Input.txt  \\
  Output.txt \\
\end{tabular}
\end{table}


\end{document}
