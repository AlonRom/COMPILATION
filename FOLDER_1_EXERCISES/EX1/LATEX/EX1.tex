\documentclass{article}
\usepackage{hyperref}
\begin{document}
\title{Exercise 1}
\maketitle

\section{Introduction}
This exercise implements an automatic checker for Linear Algebra assignments.
Currently, first year students are given a linear system $Ax=b$,
they apply the Gauss Jordan elimination algorithm, and then find the solution set.
The purpose of our exercise is to make sure those first year students
are doing their job right. For that, we will implement a tool called Lini.

\section{Lini :: Example}
Lini receives a linear system from the Professor,
and a sequence of row operations $+$ solution set from the student.
Lini outputs a detailed summary of the check results in a pdf file.

\subsection*{Linear System (Given by the Professor) :: Example}
\[
\left[
\begin{array}{ccc}
4	&	5	&	6\\
7	&	8	&	9\\
7	&	6	&	7\\
\end{array}
\right]
\left[
\begin{array}{c}
x_{1}\\
x_{2}\\
x_{3}\\
\end{array}
\right]
\begin{array}{c}
 \\
=\\
 \\
\end{array}
\left[
\begin{array}{c}
4\\
7\\
7\\
\end{array}
\right]
\]
\subsection*{Row Operations (Computed by the student) :: Example}
\begin{table}[h]
\centering
\begin{tabular}{ l }
  $R_{1} \leftarrow \frac{2}{5}R_{1}$ \\
  $R_{2} \leftarrow R_{2} - 3R_{1}$ \\
  $R_{1} \leftarrow \frac{2}{5}R_{1}$
\end{tabular}
\end{table}

\subsection*{Solution Set (Computed by the student) :: Example}
\begin{table}[h]
\centering
\begin{tabular}{ l }
  $\{(3,2,3)\} + SP(\{(1,0,1)\})$ \\
\end{tabular}
\end{table}

\section{Lini :: Details}

\subsection*{Input Format :: Linear System}
The Matrix $A$ and the vector $b$ are accommodated in a single
$m \times (n+1)$ matrix, whose rightmost column is $b$.
A matrix entry is either an integer, or a fraction.
The linear system resides in a text file,
starting with a left bracket, and ending with a right bracket.
Entries are separated with either blanks or tabs, and rows are separated with
a single semicolon. Fractions are written with either slash or backslash.
The entire system must be written as a single line in the text file:\\
\[
[ 3 ~ 4 ~ -6; 7 ~ -2/9 ~ -100; 7 ~ 0 ~ 0; -6 ~ 4/3 ~ 10]
\]

\subsection*{Input Format :: Row Operations}
The row operations sequence reside in a text file.
Each line contains a single row operation.
Table \ref{Table_CFG_Of_Row_Operations}
describes the context free grammar of row operations.
\begin{table}[h]
\centering
\begin{tabular}{ l c l }
  $S$              & $::=$ & rowOperationList                            \\
  rowOperationList & $::=$ & rowOperation ~ rowOperationList             \\
                   & $::=$ & rowOperation                                \\
  rowOperation     & $::=$ & swapRows                                    \\
                   & $::=$ & mulRow                                      \\
                   & $::=$ & mulAndAddRow                                \\
  swapRows         & $::=$ & R index leftRightArrow R index              \\
  mulRow           & $::=$ & R index leftArrow [$-$] [num] R index       \\
  mulAndAddRow     & $::=$ & R index leftArrow R index op [num] R index  \\
  index            & $::=$ & 1 ~ $|$ ~ 2 ~ $|$ ~ 3 ~ $|$ ~ 4             \\
  leftArrow        & $::=$ & $<-$                                        \\
  leftRightArrow   & $::=$ & $<->$                                       \\
  num              & $::=$ & int                                         \\
                   & $::=$ & int div int                                 \\
  int              & $::=$ & 0                                           \\
                   & $::=$ & $[1-9][0-9]^{*}$                            \\
  div              & $::=$ & $/$                                         \\
                   & $::=$ & \textbackslash                              \\
  op               & $::=$ & $+$                                         \\
                   & $::=$ & $-$                                         \\ \\ \\
\end{tabular}
\caption{
Context free grammar for the language of row operations.
\label{Table_CFG_Of_Row_Operations}}
\end{table}
In addition, the following limitations are imposed on row operations:
$(1)$ Rows can not be multiplied by zero.
$(2)$ Fractions are always with a non zero denominator.
$(3)$ The indices in rule mulRow have to be the same.
$(4)$ The first two indices in mulAndAddRow have to be the same.

\subsection*{Input Format :: Solution Set}
The solution set resides in a text file.
Whenever the system has no solutions, this text file contains a single word:
EMPTYSET. When the system has a single solution, it is written as a set 
with a single row vector $\{(3,-\frac{2}{9},200)\}$. If the system has an infinite
solution set, it should be written as a sum of two vector sets:
$\{(2,-6,100)\} ~ +$ SP$(\{(1,0,1), (6,6,0)\})$.

\section{Programming Assignment}
Write a lexer and parser for the solution set.
The exact context free grammar is missing intentionally.
You should support row vectors with sizes = $2,3,4$.
The span set size is not limited, as vectors inside it can be dependant.

\subsection*{Bonus}
Compare the given solution set to one deduced by the row operations
and see if they agree. Whenever the sets are different,
add a concrete vector that belongs to one set and not the other.

\section{Submission Guidelines}
Open an account on \href{https://github.com/}{GitHub}.
Then, visit the
\href{https://education.github.com/discount_requests/new}{academic discount page}
to enable the free creation of private repositories.
One team member should create a new private repository called COMPILATION,
and then invite other team members and me as collaborators.
My username is OrenGitHub.
Please make sure the uppermost folder (COMPILATION) contains a text file ``IDS.text" with the IDs of all team members (one ID per line).
Throughout the semester, you will create sub-folders EX1, EX2, etc. inside COMPILATION.
EX1 Should include a makefile building your source to a runnable
program called LINI.
Please support a clean target in makefile that removes LINI.
Also, to avoid the pollution of EX1,
please remove all *.o files once the target is built.
The next paragraph describes the execution of LINI.

\paragraph{Execution parameters}
LINI recevies $4$ input file names:
\begin{table}[h]
\centering
\begin{tabular}{ l }
  InputMatrix.txt   \\
  RowOperations.txt \\
  SolutionSet.txt   \\
  CheckSummary.pdf  \\
\end{tabular}
\end{table}


\end{document}
